\documentclass[12pt]{article}

\RequirePackage[utf8]{inputenc}
\RequirePackage{hyperref}
\RequirePackage{cleveref}
\RequirePackage[
    backend=biber,
    %citestyle=number,
    style=numeric
]{biblatex}
\RequirePackage{graphicx}

%% Highlight important references; from https://tex.stackexchange.com/a/149753
\usepackage[svgnames]{xcolor}
\DeclareBibliographyCategory{important}
% the colour definition
\colorlet{impentry}{Maroon}% let 'impentry' = Maroon

% Inform biblatex that all books in the 'important' category deserve
% special treatment, while all others do not
\AtEveryBibitem{%
  \ifcategory{important}%
    {\bfseries\color{impentry}}%
    {}%
  }

% Add books to 'important' category in preamble
\addtocategory{important}{%
 Wright2014,
 Crosnier2016
}


\addbibresource{report.bib}

\title{Literature review on DBLMSP and DBLMSP2 in \textit{Plasmodium}}
\date{\today}

\begin{document}
\maketitle

\subsection*{Conventions}

Throughout the report I refer to DBLMSP and DBLMSP2 as \textbf{DBs}.

All gene names use the stable IDs from \href{https://plasmodb.org/plasmo/app/}{PlasmoDB} release 53.

Important references are highlighted in the bibliography.

\section{Current state of knowledge relevant to DBs}

\subsection{Genomic location and domain structure of DBs}

In \textit{P. falciparum}, DBLMSP and DBLMSP2 are two genes located 16kbp apart on
chromosome 10 (DBLMSP: Pf3D7\_10\_v3:1,412,641..1,416,363(+); DBLMSP2:
Pf3D7\_10\_v3:1,432,498..1,434,786(+)). They are part of a set of 8 tandemly arrayed
genes spanning 32kbp on chromosome 10 that all share an N-terminal `NLRNA/G'
sequence~\cite{singh_conserved_2009}. \cite{singh_conserved_2009} call them `MSP3-like',
and proposed renaming them MSP3.1-MSP3.8. In this system MSP3.4 and MSP3.8 are the names
for what is now stably referred to as DBLMSP and DBLMSP2, respectively. Six of the eight
genes, including DBLMSP and DBLMSP2, possess a polymorphic C-terminal domain called SPAM
(\cite{MCCOLL199453}; \href{http://pfam.xfam.org/family/PF07133}{Pfam: PF07133}). In MSP3
(MSP3.1 in the MSP3-like classification), SPAM was found to mediate protein
oligomerisation~\cite{gondeau_c-terminal_2009}. The genes GLURP and LSA-1 are located
5' and 3' of the MSP3-like gene cluster, respectively. 

In contrast to the other MSP3-like genes, DBs also possess a Duffy-Binding Like (DBL)
domain (\href{http://pfam.xfam.org/family/PF05424}{Pfam: PF05424}), located between the
N-terminal NLR sequence and C-terminal SPAM. DBL is named after the initial discovery
that \textit{Plasmodium vivax} requires the so-called Duffy antigen/chemokine receptor
(DARC) on human red-blood cells for cell invasion~\cite{Miller561}. In \textit{P. vivax}
and \textit{P. knowlesi}, a gene called Duffy Binding Protein (DBP) is responsible for binding DARC. 


\subsection{DBL domains in \textit{P. falciparum}}

In addition to the DBs, the DBL domain is present in a number of other \textit{P.
falciparum} (\textit{Pf}) proteins. EBA175 was first identified as a protein mediating
RBC invasion via binding to the RBC receptor glycophorin A (GYPA)~\cite{Orlandi1992}, a
RBC surface glycoprotein (`glyco' means sugars attached). EBA175 binding requires a
sialic acid (e.g. Neu5Ac) attached to GYPA~\cite{Orlandi1992}, and antibodies raised
against EBA175 inhibit merozoite invasion of RBCs~\cite{Sim1990}. EBA175 contains two
tandemly arrayed DBL domains~\cite{Adams7085} called F1 and F2 that mediate GYPA
binding~\cite{Tolia2005}.  A number of genes paralogous to EBA175 exist that also have
tandem F1/F2 DBL domains: EBA140, EBA165, EBA181,
EBL1(\cref{fig:EBA_DBLs};~\cite{Adams2001}). EBA140 also mediates merozoite invasion of
RBCs~\cite{Mayer5222} by binding to the lycophorin C receptor linked to sialic
acid~\cite{Lobo2003}. EBA165 is functional in all \textit{Laverania} species but is
inactivated in \textit{Pf} and is thus a pseudogene~\cite{Proto2019}. EBL1 and EBA181 are
less well understood. EBL1 may bind to glycophorin B~\cite{Jaskiewicz2019}, and EBA181
binds RBCs but the receptor is unknown~\cite{Gilberger2003}.

The presence of multiple EBA paralogs is thought to mediate functional redundancy,
allowing merozoites from different strains to invade RBCs with different surface
receptors (e.g. lacking glycophorin A and B~\cite{Hadley1987}) or differently modified
surface receptors (e.g. with sialic acid removed~\cite{Mitchell1986,Stubbs2005}).
Alternative use of different EBAs for invasion was also shown to be epigenetically
mediated for EBA-140, with expression switched on or off across isogenic
subclones~\cite{Corts2007}. Another
paralogous family or proteins, the reticulocyte-binding homologs (RH), is located in the
same apical organelles as EBAs (rhoptries and micronemes). RHs are also involved in RBC invasion
and displays functional redundancy~\cite{Gunalan2012}. RHs and EBAs are thought to
function cooperatively in RBC invasion~\cite{Lopaticki2011}. Because both protein
families are recognised by the immune system, alternative invasion pathways likely
provide adaptability against immune system targeting of specific
proteins~\cite{Wright2014}.

\begin{figure}
    \includegraphics[width=\textwidth]{img/EBA_DBLs.jpg}
    \caption{DBL domains in the erythrocyte-binding like gene family. F1 and F2 are tandem 
    DBL domains homologous to \textit{P. vivax} DBL. All genes except EBL1 possess a
    conserved C-terminal cysteine rich domain. ID mapping: EBA181(\textit{jesebl}),
    EBA165(pebl), EBA140(baebl). Source:~\cite{Adams2001}.
    }
    \label{fig:EBA_DBLs}
\end{figure}

DBL domains are also found in the \textit{var} genes, a family of paralogs specific to
the subgenus \textit{Laverania}~\cite{Otto2018,otto_evolutionary_2019} (of which
\textit{Pf} is a member, but not \textit{P. vivax}).  \textit{var} genes are mostly
located in chromosome subtelomeres, with ~60 copies in the \textit{Pf}
genome~\cite{Su1995,Gardner2002}. During the ring stage of the intraerythrocytic 
cycle (IEC), \textit{Pf} expresses only one or a few \textit{var} genes at a time, and during
the schizont and trophozoite stages of the IEC, a single \textit{var} protein
product (called \textit{PfEMP1}) is found exported at the surface of the
iRBC~\cite{Chen1998}. The 5' end of \textit{var} genes contains multiple DBL and CIDR
domains mediating adherence to other cells (\cref{fig:var_DBLs}). PfEMP1 allows iRBS to
bind to uninfected red blood cells (rosetting), endothelial cell receptors in tissues,
and to platelets enabling iRBC clumping~\cite{Miller2002}. The specific combination of
DBL and CIDR domains determines the set of bound human receptors and is associated with
malaria disease severity [TODO citations].  Different domain combinations are produced
via meiotic recombination during sexual reproduction~\cite{Su1999}, but also by
non-allelic recombination~\cite{FreitasJunior2000} during mitosis~\cite{Duffy2009} (and
thus during a single infection). Recombination occurs in a structured way between
subgroups of \textit{var} genes based on genomic context (subtelomeric, central), gene
orientation and flanking non-coding sequence~\cite{Kraemer2003}.  Of all the DBL/CIDR
domains, only DBL-$\alpha$ is found in every \textit{var} gene~\cite{Smith2000}.

\begin{figure}
    \includegraphics[width=1.1\textwidth]{img/var_DBLs.jpg}
    \caption{DBL domains in the \textit{var} gene family. 
    \textit{var} genes have a N-terminal segment (NTS), transmembrane (TM) domain and
    acidic terminal segment (ATS),
    and \textit{var} types are assigned by combination of DBL and CIDR domains. The left-hand
    panel gives counts of \textit{var} types in the \textit{Pf} 3D7 genome by
    promoter type (\textit{UpsA}-\textit{UpsE}). 
    The bottom panel lists bound human proteins.  Source:~\cite{Kraemer2006}.
    }
    \label{fig:var_DBLs}
\end{figure}

PfEMP1 is recognised by the human immune system [TODO: cite]. \textit{Pf} evades immune
system recognition and clearance by switching which \textit{var} gene is expressed during
the course of infection~\cite{Roberts1992}. [TODO epigenetically mediated].

\subsection{Biological function of DBs}
\subsubsection{DB localisation and RBC binding}

DBLMSP is expressed at the late schizont and merozoite stages and localises to the merozoite surface~\cite{wickramarachchi_novel_2009}.
\cite{wickramarachchi_novel_2009} showed that COS7 (monkey-derived) cells made to
express the DBL and SPAM domains of DBLMSP bound RBCs. Binding was
abolished by treating RBCs with trypsin or neuraminidase or by adding immune sera
from DBLMSP-immunised mice. Using ELISA,~\cite{wickramarachchi_novel_2009} also showed
recombinant DBL and SPAM domains were both bound by human immune sera from
malaria-exposed, but not malaria-naive, individuals. This suggests DBLMSP is involved in
RBC binding by merozoites. However, \textit{in vitro} RBC invasion is only reduced by
25\% in the presence of high levels of anti-DBLMSP antibodies(\cite{Sakamoto2012} Fig. 4B), and RBC
invasion efficiency is not reduced in DBLMSP knock-out parasite lines~\cite{Sakamoto2012,Crosnier2016}, showing DBLMSP is not
essential for RBC invasion.

\cite{Hodder2012} find DBLMSP2 is expressed on
the surface of schizonts and merozoites using anti-DBLMSP2 antibodies, and that
full-length DBLMSP2 and the DBL domain only bind to RBCs. There are potential 
caveats to these results. In contrast to~\cite{wickramarachchi_novel_2009}, who
expressed DBLMSP on COS7 cells and assayed RBC binding,~\cite{Hodder2012} added purified
DBLMSP2 to RBCs together with \textit{Pf} post-invasion culture supernatant,
so the DBLMSP2 in pulldowns could have been in complex with other \textit{Pf} proteins
binding to RBCs.~\cite{Hodder2012} also do not find that DBLMSP binding to RBCs is
trypsin or neuraminidase sensitive, whereas~\cite{wickramarachchi_novel_2009} do.
Finally, Gavin Wright and Cecile Crosnier at the Sanger Institute could not replicate
DBLMSP2 binding to RBCs (unpublished, personal communication). 

\subsubsection{DBLMSP2 and gametocytogenesis}
DBLMSP2 transcripts are present at very low levels in bulk RNA-seq of clinical and
laboratory isolates (\cite{Otto2010} Sup. Table S3,~\cite{AmambuaNgwa2012}).
Using immunofluorescent labeling of DBLMSP2,~\cite{AmambuaNgwa2012} showed that DBLMSP2
is only present on the surface of a minority of schizonts (\cref{fig:dblmsp2_cellsurfexp}).
This is consistent with repressive chromatin factors being found at the DBLMSP2
locus: histone mark H3K9me3~\cite{LopezRubio2009} and heterochromatin protein 1
(HP1;~\cite{Fraschka2018}). These marks are not widespread in the genome and are typically
associated with clonally variant expression of virulence gene families
(e.g. \textit{var},\textit{rif},\textit{stevor},\textit{clag}~\cite{LopezRubio2009}). In
\textit{Pf}, HP1 epigenetically regulates both the single, mutually exclusive expression of
\textit{var} genes and sexual commitment via the induction of the
\textit{ap2-g} transcription factor~\cite{Brancucci2014}, which is essential for sexual
differentiation of \textit{Pf} into gametocytes~\cite{Kafsack2014}. HP1 forms a complex with
the \textit{PfGDV1} protein~\cite{Filarsky2018}, another early marker of gametocytogenesis~\cite{Eksi2012}.
Consistent with this,~\cite{Filarsky2018} found that conditional activation of \textit{PfGDV1} induces \textit{ap2-g}
expression and gametocytogenesis, likely by antagonising HP1-based repression.
Remarkably, DBLMSP2 is one of the 8 genes induced by \textit{PfGDV1}
(~\cite{Filarsky2018} Fig. 2). However, in two other studies DBLMSP2 did not appear as part of the transcriptional
signature of sexual committment defined using single-cell RNA-seq (\cite{Poran2017} Supplementary Table
3;~\cite{Brancucci2018} Supplementary File 2). It is thus unclear still whether DBLMSP2
plays a role in gametocytogenesis.

\subsubsection{DBLMSP2 and drug resistance}
DBLMSP2 has also been implicated in drug resistance. A GWAS study using SNP array
genotyping found variants in DBLMSP2 associated with resistance to the antimalarial
halofantrine~\cite{VanTyne2011}. \cite{VanTyne2011} functionally validated the
association by showing that episomal (plasmi-mediated) DBLMSP2 overexpression increases
resistance to halofantrine and structurally related compounds (mefloquine and
lumefantrine), and that \textit{Pf} strains with higher DBLMSP2 copy-number are also
more resistant. Given that DBLMSP and DBLMSP2 occur in tandem and can share
sequence~\cite{maciuca_analysis_2017}, I think WGS-based CNV quantification is 
required to validate cross-strain CNV differences in DBLMSP2.
Subsequently,~\cite{VanTyne2013} found that knocking-out DBLMSP2 reduced $IC_{50}$ (drug
concentration giving a 50\% parasite growth reduction) to the
these three antimalarials by about 50\%, and that a specific SNP (C591S) in the SPAM domain
approximately doubles $IC_{50}$ for all three antimalarials.

\begin{figure}
    \includegraphics[width=1.1\textwidth]{img/dblmsp2_cellsurfexp.png}
    \caption{DBLMSP2 is expressed in a subset of schizonts.
    (A) DAPI (parasite nuclear DNA) and DBLMSP2 immunofluorescent staining with
    anti-DBLMSP2 antibodies shows DBLMSP2 is present at the surface of a small number of
    schizonts in three independent cultured isolates (HB3, K1, D6). (B) The HB3 parasite 
    line has the highest fraction of DBLMSP2-positive schizonts.
    Source:~\cite{AmambuaNgwa2012}
    }
    \label{fig:dblmsp2_cellsurfexp}
\end{figure}
 
\subsubsection{DB complex formation}
All six `MSP3-like' proteins (which includes the DBs) localise to the merozoite
surface~\cite{singh_conserved_2009}. Despite this, all
`MSP3-like' proteins lack transmembrane domains and GPI anchors characteristic of membrane-anchored proteins. 
One `MSP3-like' protein, MSP6, was found to occur bound to GPI-mediated
membrane-anchored MSP1~\cite{Trucco2001}. This was also found for another merozoite
surface protein, MSP7~\cite{Pachebat2001}. Subsequently,~\cite{Lin2014} found that DBs
are also found in complex with MSP1. MSP1 is the most abundant merozoite surface
protein~\cite{Gilson2006} 
and is proteolytically processed into four, non-covalently linked subunits
during merozoite maturation. Upon merozoite entry into RBCs, the MSP1 complex is cleaved by the
PfSUB2 protease~\cite{Harris2005}, leaving the GPI-anchor MSP1 fragment bound to the merozoite
surface while the rest is shed~\cite{blackman_single_1990,Harris2005}. Antibodies
against MSP1 block invasion~\cite{blackman_single_1990} and are associated with
protection from malaria~\cite{Osier2008}.
AMA1, a microneme protein involved in
merozoite reorientation post RBC attachment and a major vaccine target, is also shed
following PfSUB2 cleavage~\cite{Harris2005}. Interestingly, EBA175 is also shed post-merozoite invasion
via proteolytic cleavage by the PfROM4 protein~\cite{ODonnell2006}. Subsequent work showed
shed EBA175 mediates uninfected RBC clustering via its two tandem DBL domains, 
and that RBC clustering enhances merozoite reinfection and
immune system evasion~\cite{Paing2018} (see \cref{fig:eba175_shed} for an illustrative model).

\begin{figure}
    \includegraphics[width=.7\textwidth]{img/eba175_shed.jpg}
    \caption{EBA175 shed from merozoites post RBC invasion mediates RBC clustering.
    Model from~\cite{Paing2018}. The two DBL domains from EBA175 mediate RBC clustering
    by binding to GYPA on the RBC surface.  RBC clustering reduced antibody-based
    (anti-AMA1 and anti-RH5) growth inhibition (illustrated in d) and enhanced parasite
    growth (illustrated in e).
    }
    \label{fig:eba175_shed}
\end{figure}


\subsubsection{Immune system interactions}

\cite{singh_conserved_2009} find IgGs in immune serum from people in malaria-endemic Senegal react against a
conserved C-terminal region in all six `MSP3-like' proteins, suggesting cross-reactive immunity to
all can arise.

No RBC binding partners for DBs have been identified to date, and conflicting evidence
exists regarding what kind of RBCs DBs bind (\cite{wickramarachchi_novel_2009} vs~\cite{Hodder2012}).
Further, from personal communication, Cecile Crosnier and Gavin Wright did not observe DB binding
to RBCs.

\subsection{Genetic diversity of DBs}


\clearpage

\section{Questions of interest around DBs}
$\delta$: doable
$\epsilon$: uncertain
%$\mu$: unlikely

\subsection{Biological function}
\begin{enumerate}
    \item{[$\delta$] What is the sequence and structure relationship between DBL domain in DBs and other Pf DBL domains?
        (see~\cite{Rask2010} for an analysis of DBL homology blocks, with MEME motifs,
        showing parallel between EBA-175 and VAR2CSA DBL structures (Fig.5))
        }
    \item{[$\epsilon$] Can we identify putative human protein interactants, can we dock
        DBs to them, and can we evaluate goodness of fit compared to known DBL-human
        protein interactions?}
    \item{[$\epsilon$] Can we find putative structural interactants expressed in human
        cell types likely to be in DB environment when DBs are expressed? (e.g. bone
        marrow for DBLMSP2 in early gametocytes}
    \item{[$\delta$] How often are DBs not functional (nonsense mutations) in natural
        pops, and is there a diff. between DBLMSP and DBLMSP2? (Note 
        premature stop codons are known in DBLMSP: \cite{Tetteh2009};
        ~\cite{ochola_allele_2010} Table 1)}
    \item{[$\delta$] What is the phylogenetic distribution of DBs?
        }
\end{enumerate}

\subsection{Location of polymorphisms}
\begin{enumerate}
    \item{[$\delta$] Do we see conserved/hypervariable sites at the DNA level?}
    \item{[$\delta$] Can we map polymorphisms onto DB structures and identify
        structurally conserved or variable regions? (BioStruct can do
        this~\cite{Guy2018a}, see~\cite{Guy2018b} for examples in \textit{Pf}}
\end{enumerate}
\subsection{Gene conversion and recombination}
\begin{enumerate}
    \item{[$\epsilon$] Can we identify and quantify gene conversion events between DBs?
        (Note, could maybe also do this in tandemly arrayed genes clag 3.1 and clag 3.2,
        this has been observed in~\cite{Iriko2008})
        }
\end{enumerate}
\subsection{Drug resistance}
\begin{enumerate}
    \item{[$\delta$] Is the drug resistance SNP in DBLMSP2 SPAM domain found in DBLMSP
        sequences (or other `MSP3-like' proteins with SPAM domains)?
        }
\end{enumerate}



\clearpage
\printbibliography

\end{document}
